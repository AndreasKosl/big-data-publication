% Compile this file with pdflatex to create the list
% of cited papers in the aux file: 'pdflatex latex-examples'
% Then compile the aux file with bibtex to extract the information
% about the cited papers from your bibtex-database: 'bibtex latex-examples'.
% Then you must recompile the tex file with pdflatex to add the list of
% references to the pdf file.
% Because the bibliography is only included in the end of the tex file
% pdflatex did not know the number of each citation during the last
% compilation and therefore added only '[?]' where you cited something.
% You have to compile it once more to finish the pdf document.

\documentclass[runningheads]{llncs}

\usepackage[pdftex]{graphicx}
\usepackage[plainpages=false, pdfpagelabels, bookmarks,  colorlinks=false, % Hyperlinks in PDF output mit blauen Rahmen
               linkbordercolor={0 0 1}, filebordercolor={0 0 1}, citebordercolor={0 0 1},
              menubordercolor={0 0 1}, urlbordercolor={0 0 1}]{hyperref}


\title{Big Data Analytics for Business Intelligence in Vertical Industries}
\subtitle{Seminar Big Data WS 2015/16}
\author{Anton Okolnychyi, Andreas Koslowski}
\institute{RWTH Aachen University, 52056 Aachen, Germany\\
\{anton.okolnychyi, andreas.koslowski\}@rwth-aachen.de}

\begin{document}

\maketitle

\begin{abstract}
to be done
\end{abstract}

\section{Introduction}
The business world is going through a revolution induced by the use of data to control decision-making and to perform analytics. A major reason for the business analytics revolution is the rapid proliferation of the amount of data available to be analysed \cite{Gopalkrishnan}. Tasks in modern analytics require a huge computing power, storage capacity, appropriate information technologies which are needed to gather, analyse and retrieve an asset from data. This becomes possible due to the constant evolution of the corresponding software and hardware techniques. The availability of computing power and data storage capacity have expanded at an exponential pace, and this trend seems unlikely to abate any time soon \cite{NYTIMES}.  

Big companies understand the value of their data and possible outcomes of usage of data analysis in adjusting and defining their business strategy.  Some of companies try to make use of opinions expressed by customers in order to improve the overall customer experience. Others analyse data to find patterns in customer's behaviour which can be used later to predict future needs and purchases. Data is a strategic asset in making recommendations. Often, data is analysed in order to fine-tune the enterprise itself, with analytical insights used to refine internal processes, promote safety, and pinpoint operational issues the resolution of which can drive up efficiency, profitability, and competitive positioning \cite{Guszcza}.
This is especially true in the vertical industries, where it is critical to adapt to the needs of the customers and predict the trends of each specific market. By analysing the data the industries gathers, it is much easier to adapt to current trends and they are able to offer the customers complete business solutions.

This paper provides an overview of how Big Data is used in the industry and clarifies how Big Data and it's use will change the Business Intelligence, particularly in the vertical markets. It'll compare how the different vertical industries are dealing with the topic of Big Data Analytics and what strategies are adopted while doing so.

\section{Big Data Analytics}
Big data analytics is the use of advanced analytic techniques against very large, diverse data sets that include different types such as structured/unstructured and streaming/batch, and different sizes from terabytes to zettabytes \cite{IBM}. The main aim of big data analytics is to assist organisations in making smart business decisions by enabling data scientists and other analytics professionals to analyse large volumes of data that  was previously inaccessible, unusable or just untapped.  With a help of advanced methods and tools, such as text analytics, machine learning, data mining, statistics, and natural language processing, companies are able to perform analysis of previously untapped data sources to retrieve new insights which lead to better and more informed decisions. Just having a lot of data does not give anything. Only using the correct analytics one can extrapolate powerful insights. Amir Gandomi, Murtaza Haider say in \cite{ELS}, that Big data is worthless in a vacuum. Its potential value is unlocked only when leveraged to drive decision making. To enable such evidence-based decision making, organisations need efficient processes to turn high volumes of fast-moving and diverse data into meaningful insights. Charter Global in \cite{CHARTER} defines 3 types of analytics: Descriptive, Predictive and Prescriptive. 

Descriptive Analytics is the simplest kind of analytics. It allows you to retrieve smaller and more useful bits of information from huge data sets. Alternatively, it may provide a summary of what happened. Social media analytics is one of examples of Descriptive Analytics. It refers to the analysis of structured and unstructured data from social media channels. User-generated content (e.g., sentiments, images, videos, and bookmarks) and the relationships and interactions between the network entities (e.g., people, organizations, and products) are the two sources of information in social media \cite{ELS}. 

Predictive analytics is the next step up in data reduction. It utilizes a variety of statistical, modeling, data mining and machine learning techniques to study recent and historical data, thereby allowing analysts to make predictions about the future\cite{CHARTER}. Predictive Analytics is the use of historical data to forecast on consumer behaviour and trends \cite{TALLIN}. This type of analysis makes use of the statistical models and machine learning algorithms to identify patterns and learn from historical data \cite{MIS}. At its core, predictive analytics seek to uncover patterns and capture relationships in data. Predictive analytics techniques are subdivided into two groups. Some techniques, such as moving averages, attempt to discover the historical patterns in the outcome variable(s) and extrapolate them to the future. Others, such as linear regression, aim to capture the interdependencies between outcome variable(s) and explanatory variables, and exploit them to make predictions \cite{ELS}.

Prescriptive Analytics does not predict only one future, it rather predicts "multiple futures" based on the decision-maker’s potential actions. Each decision is associated with a likely outcome. This can be used to choose the better action. Thomas H. Davenport and Jill Dyche in \cite{DAVENPORT} say, that there is an increased emphasis on prescriptive analytics. It provides a high-level of operation benefits for organisations and it places a premium on high-quality planning and execution. 

\bibliographystyle{splncs} % other possible styles: plain, unsrt, abbrv, alpha
\bibliography{Seminar-Bib}

\end{document}
