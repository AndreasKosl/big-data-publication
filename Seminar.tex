% Compile this file with pdflatex to create the list
% of cited papers in the aux file: 'pdflatex latex-examples'
% Then compile the aux file with bibtex to extract the information
% about the cited papers from your bibtex-database: 'bibtex latex-examples'.
% Then you must recompile the tex file with pdflatex to add the list of
% references to the pdf file.
% Because the bibliography is only included in the end of the tex file
% pdflatex did not know the number of each citation during the last
% compilation and therefore added only '[?]' where you cited something.
% You have to compile it once more to finish the pdf document.

\documentclass[runningheads]{llncs}

\usepackage[pdftex]{graphicx}
\usepackage[plainpages=false, pdfpagelabels, bookmarks,  colorlinks=false, % Hyperlinks in PDF output mit blauen Rahmen
               linkbordercolor={0 0 1}, filebordercolor={0 0 1}, citebordercolor={0 0 1},
              menubordercolor={0 0 1}, urlbordercolor={0 0 1}]{hyperref}


\title{Big Data Analytics for Business Intelligence in Vertical Industries}
\subtitle{Seminar Big Data WS 2015/16}
\author{Anton Okolnychyi, Andreas Koslowski}
\institute{RWTH Aachen University, 52056 Aachen, Germany\\
\{anton.okolnychyi, andreas.koslowski\}@rwth-aachen.de}

\begin{document}

\maketitle

\begin{abstract}
to be done
\end{abstract}

\section{Introduction}
The business world is going through a revolution induced by the use of data to control decision-making and to perform analytics. A major reason for the business analytics revolution is the rapid proliferation of the amount of data available to be analysed \cite{Gopalkrishnan}. Tasks in modern analytics require a huge computing power, storage capacity, appropriate information technologies which are needed to gather, analyse and retrieve an asset from data. This becomes possible due to the constant evolution of the corresponding software and hardware techniques. The availability of computing power and data storage capacity have expanded at an exponential pace, and this trend seems unlikely to abate any time soon \cite{NYTIMES}.  

Big companies understand the value of their data and possible outcomes of usage of data analysis in adjusting and defining their business strategy.  Some of companies try to make use of opinions expressed by customers in order to improve the overall customer experience. Others analyse data to find patterns in customer's behaviour which can be used later to predict future needs and purchases. Data is a strategic asset in making recommendations. Often, data is analysed in order to fine-tune the enterprise itself, with analytical insights used to refine internal processes, promote safety, and pinpoint operational issues the resolution of which can drive up efficiency, profitability, and competitive positioning \cite{Guszcza}.
This is especially true in the vertical industries, where it is critical to adapt to the needs of the customers and predict the trends of each specific market. By analysing the data the industries gathers, it is much easier to adapt to current trends and they are able to offer the customers complete business solutions.

This paper provides an overview of how Big Data is used in the industrie and clarifies how Big Data and it's use will change the Business Intelligence, particularly in the vertical markets. It'll compare how the different vertical industries are dealing with the topic of Big Data Analytics and what strategies are adopted while doing so.

\bibliographystyle{splncs} % other possible styles: plain, unsrt, abbrv, alpha
\bibliography{Seminar-Bib}

\end{document}
